\documentclass{article}

% Recommended, but optional, packages for figures and better typesetting:
\usepackage{microtype}
\usepackage{graphicx}
\usepackage{subfigure}
\usepackage{booktabs} % for professional tables
\usepackage{amsmath}
\usepackage{amsfonts}

% hyperref makes hyperlinks in the resulting PDF.
% If your build breaks (sometimes temporarily if a hyperlink spans a page)
% please comment out the following usepackage line and replace
% \usepackage{icml2018} with \usepackage[nohyperref]{icml2018} above.
\usepackage{hyperref}

% Attempt to make hyperref and algorithmic work together better:
\newcommand{\theHalgorithm}{\arabic{algorithm}}

% Use the following line for the initial blind version submitted for review:
%\usepackage{icml2018}

% If accepted, instead use the following line for the camera-ready submission:
\usepackage[accepted]{icml2018}

% The \icmltitle you define below is probably too long as a header.
% Therefore, a short form for the running title is supplied here:
%\icmltitlerunning{}

\begin{document}

\twocolumn[
\icmltitle{Genetic Algorithm for Weighted 3-SAT Problem}

% It is OKAY to include author information, even for blind
% submissions: the style file will automatically remove it for you
% unless you've provided the [accepted] option to the icml2018
% package.

% List of affiliations: The first argument should be a (short)
% identifier you will use later to specify author affiliations
% Academic affiliations should list Department, University, City, Region, Country
% Industry affiliations should list Company, City, Region, Country

% You can specify symbols, otherwise they are numbered in order.
% Ideally, you should not use this facility. Affiliations will be numbered
% in order of appearance and this is the preferred way.
\icmlsetsymbol{equal}{*}

\begin{icmlauthorlist}
\icmlauthor{Ond\v{r}ej Podsztavek}{fit}
\end{icmlauthorlist}

\icmlaffiliation{fit}{
Faculty of Information Technology, Czech Technical University in Prague,
Prague, Czech Republic
}

\icmlcorrespondingauthor{Ond\v{r}ej Podsztavek}{podszond@fit.cvut.cz}

% You may provide any keywords that you
% find helpful for describing your paper; these are used to populate
% the "keywords" metadata in the PDF but will not be shown in the document
\icmlkeywords{genetic algorithm, SAT problem}

\vskip 0.3in
]

% this must go after the closing bracket ] following \twocolumn[ ...

% This command actually creates the footnote in the first column
% listing the affiliations and the copyright notice.
% The command takes one argument, which is text to display at the start of the footnote.
% The \icmlEqualContribution command is standard text for equal contribution.
% Remove it (just {}) if you do not need this facility.

\printAffiliationsAndNotice{}  % leave blank if no need to mention equal contribution
%\printAffiliationsAndNotice{\icmlEqualContribution} % otherwise use the standard text.

\begin{abstract}
TODO abstract.
\end{abstract}

\section{Goal}

This work aims to use a genetic algorithm to solve as wide spectrum of
weighted 3-SAT problems defined in next section~\ref{problem} as possible.
Therefore, the task consists of finding appropriate setting of genetic
algorithm's parameters.

\section{Weighted 3-SAT Problem}
\label{problem}

This section defines weighted 3-SAT problem which is of same difficulty
as Boolean satisfiability (SAT) problem.
SAT problem is proven to be NP-complete problem
which means that all problems in the complexity class NP are at most
as difficult to solve as SAT.~\cite{cook1971}

Given a Boolean formula $F: \mathbb{B}^n \to \mathbb{B}$,
$\mathbb{B} = \{0, 1\}$ in conjunctive normal form with $n$ variables
$X = (x_1, x_2, \dots, x_n) \in \mathbb{B}^n$
where each clause has exactly 3 literals.
Moreover, there are integer weights
$W = (w_1, w_2, \dots, w_n) \in \mathbb{N}$.
Find values $Y = (y_1, y_2, \dots, y_n) \in \mathbb{B}^n$ of variables from $X$
such that $F(Y) = 1$
and sum of weight of variable that has value 1 is maximal.

\subsection{Instances}

For experimental purpose of this work uniform random 3-SAT instances
from SATLIB Benchmark Problems~\cite{hoos2000} are going to be used.
Since genetic algorithms is \textit{incomplete algorithm}\footnote{Incomplete
algorithms for SAT are able to decide if a given formula is satisfiable by
finding a truth assignment
but they cannot detect contradictory formulas
thus ascertain unsatisfiability.~\cite{ellerweg2004}}
only satisfiable instances are included.

The instances are in DIMACS format
which means that they do not contain weights so they are uniformly randomly
generated from set of natural numbers less or equal than TODO.
The upper bound should affect the fitness of an individual
therefore such arbitrary number was chosen.

In order to measure performance of a genetic algorithm
there is a need differently difficult formulas.
A number of paper have claimed that most randomly generated instances are
not that difficult.
But it was empirically found that 3-SAT difficulty depends on ratio of
number of clauses and variables $\frac{m}{n}$.
The most difficult instances has about 4.3 times more clauses than variables.
\cite{selman1996}

Instances from SATLIB Benchmark Problems have usually the highest difficulty.
For example, there are instances with 218 clauses and 50 variables
giving clauses-to-variables ratio of 4.36.
Creation of easier instances can be done by leaving out clauses
which is going to lower the ratio and not affect satisfiability.

TODO discuss the testbed used.

\section{Genetic Algorithm}

The idea of genetic algorithms is to evolve a population of solutions
to a given problem using operators inspired by natural genetic variation
and natural selection.
\cite{mitchell1996}

There is no rigorous definition of genetic algorithm but most methods
have at least in common populations of chromosomes,
selection according to fitness, crossover to produce new offspring
and random mutation of new offspring.
\cite{mitchell1996}

The term \textit{chromosome} refers to a solution to a problem
which is encoded as a bit string
and can be thought of as points in the search space of solutions.
The \textit{genes} are either single or short blocks bits
that encode a particular element of the solution.
An \textit{allele} is 0 or 1 in bit string.
Mostly genetic algorithms employ \textit{single-chromosome haploid} individuals.
The \textit{genotype} of an individual is the configuration of bits
in its chromosome.
\cite{mitchell1996}

The simplest form of genetic algorithm involves three types of operators.
\textit{Selection} in operator which selects chromosomes in the population for
reproduction.
The fitter the chromosome the more it is likely to be selected.
\textit{Crossover} randomly chooses a position in a chromosome
and exchange the subsequences before and after the position between
two chromosomes to create two new offspring.
\textit{Mutation} randomly flips some bits in a chromosome.
\cite{mitchell1996}

The algorithm processes population of chromosomes successively replacing
one population with another (see algorithm~\ref{alg:ga}).
Genetic algorithm requires a fitness function that assigns a score to
each chromosome in the current population.
The chromosome's fitness depends on how well the chromosome solves given
problem.
\cite{mitchell1996}

\begin{algorithm}[hb]
\caption{Genetic Algorithm~\cite{eiben2003}}
\label{alg:ga}
\begin{algorithmic}
\STATE initialize population with random individuals
\STATE evaluate each individual
\WHILE{terminal condition is not satisfied}
\STATE select parents
\STATE recombine pairs of parents
\STATE mutate the resulting offspring
\STATE evaluate new individuals
\STATE select individuals for next generation
\ENDWHILE
\end{algorithmic}
\end{algorithm}

\section{Implementation}

The implementation is in Python 3.6 programming language
and is based on DEAP framework~\cite{fortin2012}.

Chromosomes representation for weighted 3-SAT problem is straight forward.
A chromosome is represented as Python's list of $n$ numbers 0 or 1.
If there is 1 resp. 0 at position $i$ it means that $x_i$ is set to 1 resp. 0.

The skeleton of implemented genetic algorithm is the same as
in algorithm~\ref{alg:ga}.
Population size is fixed through all generation.

In order to stop the evolution of population advantage of
\textit{early stopping} is taken.
This mechanism works based on average fitness of population increase.
If population's average fitness does not increase by a certain amount
during given number of generation the evolution is terminated.

Last commonly used method applied is \textit{elitism}.
This method puts chosen number of elite individuals with best fitness
from previous generation to next generation.

\section{Experiments}

TODO cite~\cite{gottlieb2002} and \cite{ellerweg2004}.

Experiments were carried out on Intel Core i5-3337U CPU (1.80GHz).

\subsection{Fitness Functions}

TODO MAX-SAT fitness function from~\cite{dejong1989}:

$$f_{\text{MAX-SAT}}(Y) = c_1(Y) + \dots + c_m(Y),$$

where $m$ is the number of clauses and $c_i(x)$ is the truth value of the $i$th
clause.

The value of fitness function $f$ is the sum of weights of variables of value 1
if the formula is satisfied ($F(Y) = 1$)
else it is negative number of unsatisfied clauses:

$$
f(Y) = 
\begin{cases} 
    \sum_{i = 1}^m y_i w_i & \text{if } F(Y) = 1, \\
    f_{\text{MAX-SAT}}(Y) - m & \text{if } F(Y) = 0,
\end{cases}
$$

where TODO.

\subsection{Selection Method}

\subsection{Parameters}

\section{Conclusion}

% In the unusual situation where you want a paper to appear in the
% references without citing it in the main text, use \nocite
%\nocite{langley00}

\bibliography{sat-report}
\bibliographystyle{icml2018}

\end{document}

% This document was modified from the file originally made available by
% Pat Langley and Andrea Danyluk for ICML-2K. This version was created
% by Iain Murray in 2018. It was modified from a version from Dan Roy in
% 2017, which was based on a version from Lise Getoor and Tobias
% Scheffer, which was slightly modified from the 2010 version by
% Thorsten Joachims & Johannes Fuernkranz, slightly modified from the
% 2009 version by Kiri Wagstaff and Sam Roweis's 2008 version, which is
% slightly modified from Prasad Tadepalli's 2007 version which is a
% lightly changed version of the previous year's version by Andrew
% Moore, which was in turn edited from those of Kristian Kersting and
% Codrina Lauth. Alex Smola contributed to the algorithmic style files.
